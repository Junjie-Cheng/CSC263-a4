\documentclass[11pt, answers]{exam}
\renewcommand{\baselinestretch}{1.05}
\usepackage{amsmath,amsthm,verbatim,amssymb,amsfonts,amscd, graphicx}
\usepackage{graphics}

\usepackage{afterpage}
\usepackage{caption}

\usepackage{fancybox}

\usepackage{clrscode3e}

\topmargin0.0cm
\headheight0.0cm
\headsep0.0cm
\oddsidemargin0.0cm
\textheight23.0cm
\textwidth16.5cm
\footskip1.0cm
\theoremstyle{plain}
\newtheorem{theorem}{Theorem}
\newtheorem{corollary}{Corollary}
\newtheorem{lemma}{Lemma}
\newtheorem{proposition}{Proposition}
\newtheorem*{surfacecor}{Corollary 1}
\newtheorem{conjecture}{Conjecture}  
\theoremstyle{definition}
\newtheorem{definition}{Definition}

 \begin{document}
 


\title{CSC263: Assignment 4}
\date{March 16th, 2017}
\author{Ci Zhang, Xinyi Gong, Junjie Cheng}
\maketitle

\unframedsolutions

\begin{questions}
\question
%Question1
\begin{solution}Written by Ci Zhang, revised by Xinyi Gong and Junjie Cheng


\end{solution}

\question
%Question2
\begin{solution}Written by Xinyi Gong, revised by Ci Zhang and Junjie Cheng
$B = c + {\frac{3}{2}}d$. \newline
We will respectively prove it is an upper bound by Aggregate and Accounting method first.\newline
\newline
\textbf{Aggregate method}

This amortized cost can be separated into 2 parts. The \textbf{\emph{fixed cost}} and the \textbf{\emph{cost to flip the digits}}.

\begin{enumerate}

\item
The average fixed cost of each operation, and therefore the amortized cost per operation, is $\frac{cn}{n} = c$.
\item
Suppose k is the total number of digits. 
We use an array D[0...k - 1] of bits, where D.length = k, as the display. A binary number x that is stored in the display has its lowest-order bit in D[0] and its highest-order bit in D[k - 1], so that $x = \sum_{i=0}^{k-1} A[i]3^i$. Initially, $x = 0$, and thus $D[i] = 0$ for $i = 0, 1,..., k - 1$. To add 1 (modulo 3k) to the value in the display, we use the following procedure.

INCREMENT(D)

1\qquad i = 0

2\qquad while i $<$ D.length and D[i] == 2

3\qquad\qquad D[i] = 0

4\qquad\qquad i = i + 1

5\qquad if i $<$ D.length

6\qquad\qquad D[i] += 1

D[0] does flip each time INCREMENT is called.
The next bit up, D[1], flips every 3 times: a sequence of n INCREMENT operations on an initially zero counter causes D[1] to flip $\floor{\frac{n}{3}}$ times. Similarly,
bit D[2] flips only every ninth time, or $\floor{\frac{n}{9}}$ times in a sequence of n INCREMENT
operations. In general, for $i = 0, 1, ..., k - 1$, bit D[i] flips $\floor{\frac{n}{3^i}}$ times in a
sequence of n INCREMENT operations on an initially zero counter. For $i = k$,
bit D[i] does not exist, and so it cannot flip. The total number of flips in the
sequence is thus

$\sum_{i=0}^{k-1}\floor{\frac{n}{3^i}} < n\sum_{i=0}^{\infty}{\frac{1}{3^i}} = {\frac{3n}{2}}$

The worst-case time for a sequence of n INCREMENT operations
on an initially zero counter is therefore 3n. The average cost of each operation,
and therefore the amortized cost per operation, is $\frac{3dn}{2n} = {\frac{3}{2}}d$.
\end{enumerate}
\qquad So the overall amortized cost is $B = c + {\frac{3}{2}}d$. \newline
\newline
\textbf{Accounting method}

Let us charge an amortized cost of 3d dollars to set a bit from 0 to 1.When a bit is set, we use d dollars (out of the 3d dollars charged) to pay for the actual setting of the bit, and we place d dollars on the bit as credit to be used later when we flip the bit from 1 to 2, and d dollars on the bit as credit to be used later when we flip the bit from 2 back to 0. At any point in time, every 1 in the display has 2d dollars of credit on it, and thus we can charge nothing to change a bit from 1 to 2 and reset a bit to 0; we just pay for the change from 1 to 2 and the reset with the dollars bill on the bit. Now we can determine the amortized cost of INCREMENT. The cost of resetting the bits within the while loop is paid for by the dollars on the bits that are reset. The INCREMENT procedure sets a bit from 0 to 1 every 2 times when line 6 is executed, and therefore the amortized cost of an INCREMENT operation is at most ${\frac{1}{2}} \times 3d = {\frac{3}{2}}d$ dollars. The number of 1s in the counter never becomes negative, and thus the amount of credit stays nonnegative at all times. Thus, for n INCREMENT operations, the total amortized cost is c + ${\frac{3}{2}}$d, which bounds the total actual cost.
\newline
\newline
\textbf{Prove ${\frac{3d}{2}}$ is the smallest among the list above}

We will prove it is the smallest one by giving a counter example such that $A(n) > {\frac{17d}{12}}$. When $n = 9, A(n) = {\frac{14d}{9}} = {\frac{56d}{36}} > {\frac{17d}{12}} = {\frac{51d}{36}}$
\end{solution}


\question
%Question3
\begin{solution}Written by Junjie Cheng, revised by Ci Zhang and Xinyi Gong


We are charging 12 for each \proc{Insert} operation and 0 for each \proc{Diminish} operation.

The credit invariant is, after any operation, the credit balance is always no less than $12$ times the number of elements in the set (i.e. credit balance $\ge 12m$ always hold after any operation).

Proof: We define $k$ to be the number of operations performed. We can prove this by induction on $k$. 

Base case: $k=1$

Before the first operation is performed, the credit balance and the number of element in the set (i.e. $n$) are both $0$. If the first operation is \proc{Insert}, then we charge $12$ but do not make any comparisons. So after the operation there will be $1$ element, with the credit balance $12$. Credit invariant holds.

If the first operation is \proc{Diminish}, then according to the algorithm, since the set is empty, no pairwise comparisons between the elements of the set will be made. $0$ credit is charged and no comparisons are made; $n$ remains $0$ as there is no insertion. The credit invariant also holds.

Induction step: We assume the credit invariant holds for $k=p-1$, and now we are showing it also holds for $k=p$.

By induction hypothesis, the credit balance is now no less than $12m$, where $n$ is a non-negative number. Let's use $c$ to denote credit balance, we have $c_{p-1} \ge 12n_{p-1}$.

If the $p^{th}$ operation is \proc{Insert}, then $n$ will increment by one, and credit balance $c$ is increased by $12$ since no pairwise comparisons between the elements of the set is made. So, we have $$c_p = c_{p-1} + 12 \ge 12n_{p-1} + 12 = 12(n_{p-1}+1) = 12n_p$$. The credit invariant holds.

If the $p^{th}$ operation is \proc{Diminish}, then no credit is charged and we will use the credit balance to perform the comparison, meanwhile, the set size $n$ is also decreased.

The first step of the algorithm for \proc{Diminish} performs at most $5n_{p-1}$ pairwise comparisons and the second step performs $n_{p-1}$ comparisons (one comparison is made between each element in the set and $m$ obtained in step one). The third and fourth step does not involve any comparisons. So in total, $5n_{p-1}+n_{p-1}=6n_{p-1}$ pairwise comparisons between the elements in the set are made. The credit balance now becomes $c_p=c_{p-1}-6n_{p-1} \ge 12n_{p-1} - 6n_{p-1} = 6n_{p-1}$. 

On the other hand, the number of elements in the set is diminished to $\lfloor \frac{n_{p-1}}{2} \rfloor$. So after the operation the number of elements left in the set is $n_p = \lfloor \frac{n_{p-1}}{2} \rfloor \le \frac{n_{p-1}}{2}$.

So the new relationship between $n$ and $c$ is $$c_p \ge 6n_{p-1} = 12\frac{n_{p-1}}{2} \ge 12 n_p$$. The credit invariant still holds.

Thus, the credit invariant is $c \ge 12n$.

According to the credit invariant, the credit balance is always non-negative (since the balance is always no less than a multiple of non-negative number), i.e, we have always charged enough up front to pay for \proc{Diminish} operations. Thus, for any sequence of $k$ \proc{Insert} and \proc{Diminish} operations, the total amortized cost is an upper bound on the total actual cost. Also note that the total amortized cost is $12$ times the number of \proc{Diminish} operations, the amortized cost is in $O(k)$. Thus, for each operation, the amortized cost is $\frac{O(k)}{k} = O(1)$.


\end{solution}

\end{questions}



\end{document}
